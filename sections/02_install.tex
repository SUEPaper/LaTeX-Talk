\section{\LaTeX{} 安装}

\begin{frame}{选择发行版}
    \begin{itemize}
    \item \TeX{} 发行版distribution

        \begin{itemize}
        \item 引擎、宏包、字体、文档的综合体
        \item 类比 Visual Studio
        \item \TeX{} Live、Mac\TeX{}、W32\TeX{}、MiK\TeX{} 等
        \end{itemize} \pause

    \item \TeX{} Live \link{https://www.tug.org/texlive}

        \begin{itemize}
        \item 官方维护,首选,跨平台
        \item Mac\TeX{} ≈ macOS 下的 \TeX{} Live
        \item 缺点:完整版体积大7GB+、每年需重装
        \end{itemize}

    \item MiK\TeX{} \link{https://miktex.org}

        \begin{itemize}
        \item 由 Christian Schenk 维护(是个狠人)
        \item 宏包随用随装
        \item 缺点:部分细节与 \TeX{} Live 不兼容、网络问题
        \end{itemize} \pause

    \item \alert{不要安装 \CTeX{} 套装!}

        \begin{itemize}
        \item \alert{存在严重 bug,并且完全过时(2012年已经停止维护)。}
        \end{itemize}
    \end{itemize}
\end{frame}


\begin{frame}{选择本地编辑器}
  LaTeX 也是纯文本文件,后缀名为 .tex,可以用任何文本编辑器编写

    \begin{itemize}
      \item<+-> 专用型
    
        \begin{itemize}
          \item TeXworks:\TeX{} Live 自带 \faWindows{} \faApple{} \faLinux{}
          \item \emph{TeXStudio}:功能丰富,对新手友好 \faWindows{} \faApple{} \faLinux{}
          \item TeXShop:Mac\TeX{} 自带 \faApple{}
          \item WinEdt:功能丰富,收费 \faWindows{}
        \end{itemize}
    
      \item<+-> 通用型
    
        \begin{itemize}
          \item \emph{Visual Studio Code}:借助插件 \pkg{LaTeX Workshop} + \pkg{LaTeX Utilities}
          \item Sublime Text:收费
          \item Vim:q、q!、wq、wq!、...???
          \item Emacs:ctrl-s、ctrl-c ctrl-x、...???
        \end{itemize}

    \end{itemize}
\end{frame}

\begin{frame}{不想安装?}
  \begin{itemize}
  \item 云端服务更好用…吗?
  \item 免去安装、升级等一系列烦恼,可以多人协作
  \item 版本控制、模板市场
  \end{itemize} \pause

  \begin{itemize} \small
  \item Overleaf:
      \href{https://cn.overleaf.com}{\textcolor[HTML]{138a07}{Overleaf} \faLink}

      \begin{itemize}
      \item 模板丰富
      \item 用户支持很好
      \pause
      \item 注意网络环境(咳咳)
      \end{itemize}
  \end{itemize}

\end{frame}

\begin{frame}{引擎}
  \begin{itemize}
    \item 引擎即编译器,是编译源代码生成文档的排版引擎
      \begin{itemize}
        \item 有 TeX、pdfTeX、XeTeX、LuaTeX 等
      \end{itemize}
    \item 有两种语言的格式,分别为 plainTeX(最初版本)和 LaTeX
    \item 编译指令即编译源代码生成文档的命令,根据引擎和格式来选择
    \begin{table}[]
      \begin{tabular}{|c|c|c|c|}
      \hline
       引擎 & 目标格式 & 使用plainTex  & 使用 LaTeX  \\ \hline
       TeX & DVI & tex &  \\ \hline
       pdfTeX & DVI & etex &  latex \\ \hline
       pdfTeX & PDF & pdftex & 	pdflatex \\ \hline
       XeTeX & PDF & xetex & xelatex \\ \hline
       LuaTeX & PDF & luatex & lualatex \\ \hline
      \end{tabular}
    \end{table}
  \end{itemize}
\end{frame}

\begin{frame}{编译指令}
  \begin{itemize}
    \item 纯文本文件,后缀名为 \texttt{.tex}(类比 Markdown 的 \texttt{.md})
    \pause
    \item 选择正确的编译指令
        \begin{itemize}
            \item 例如 \texttt{xelatex main.tex}
            \item 文中含有交叉引用、参考文献、目录等情况,需要多次编译
            \item \texttt{latexmk} 秒了
        \end{itemize}
    \pause
    \item VSCode:LaTeX Workshop 插件秒了
    \item Overleaf 秒了
  \end{itemize}
\end{frame}


\begin{frame}{推荐安装}
    \begin{itemize}
      \item<+-> 本地版最佳实践
    
        \begin{itemize}
            \item + MiK\TeX{}
            \item + Visual Studio Code (LaTeX Workshop 等插件)
            \item + git(代码管理工具)
            \item + Github(全世界最大的程序员交友网站)
        \end{itemize}
      
      \item<+-> 网络版最佳实践
        \begin{itemize}
          \item 在线编辑器 \href{https://cn.overleaf.com}{\textcolor[HTML]{138a07}{Overleaf} \faLink} 编写
        \end{itemize}

      \item<+-> \href{https://suepaper.github.io/math201/docs/latex/}{\textcolor[HTML]{138a07}{保姆级手把手的教程} \faLink}

    \end{itemize}
\end{frame}

