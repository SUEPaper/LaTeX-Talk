\section{常用环境}

\subsection{表格}
\begin{frame}[fragile]{表格}
  \begin{itemize}
      \item \texttt{tabular} 环境,一般包裹在 \texttt{table} 环境中变成浮动体
      \mintinline{tex}{\begin{tabular}}\texttt{\small \{\textit{column spec}\}}
      \begin{itemize}
          \item \texttt{l} / \texttt{c} / \texttt{r}:左/中/右对齐
          \item \texttt{|}:竖线分隔;\texttt{@\{\}}:去除列间距;\texttt{@\{...\}}:自定义列间内容
          \item \texttt{\*\{\textit{num}\}\{\textit{col}\}}:重复 \texttt{\textit{num}} 次 \texttt{\textit{col}} 列格式
      \end{itemize}
      \item 在表格内容中 \texttt{\&} 分隔列,\mintinline{tex}{\\} 换行,\mintinline{tex}{\hline} 画横线
      \item \texttt{booktabs} 宏包提供三线表式样
      \item 也可以合并单元格、拆分单元格等更复杂的操作
      \item 推荐使用
      \begin{itemize}
          \item \href{https://www.tablesgenerator.com/}{Tables Generator} 网站
          \item \href{https://www.latex-tables.com/}{Tables Editor} 网站
          \item 表格真的太难写了
      \end{itemize}
  \end{itemize}
\end{frame}

\subsection{代码}
\begin{frame}[fragile]{代码环境}
  \begin{itemize}
      \item 行内代码使用 \mintinline{tex}{\verb}{\small \texttt{<\textit{delim}>...<\textit{delim}>}}
      \begin{itemize}
          \item 区分于正常的 \texttt{\{...\}}
          \item \texttt{\textit{delim}} 可以是除了星号 \texttt{*} 的任意字符
          \item \mintinline{tex}{\verb*} 命令表示显示空格
      \end{itemize}
      \item 代码环境:\texttt{verbatim}
      \begin{itemize}
          \item 默认使用等宽字体,不解析 \LaTeX{} 命令
          \item 内容中的特殊字符都不需要转义
      \end{itemize}
      \item \texttt{listings} 宏包提供代码高亮
      \item \texttt{minted} 功能更强大,需要安装 Python 依赖,本教程的代码环境都是\texttt{minted}
  \end{itemize}
  \pause
  \begin{columns}
      \begin{column}{0.4\textwidth}
          \begin{minted}[frame=lines,framesep=2mm,baselinestretch=1.2,bgcolor=LightGray,fontsize=\scriptsize]{tex}
  \verb|\LaTeX ^_^|
  and
  \verb*`printf("Hello, world!\n");`
          \end{minted}
      \end{column}
      \begin{column}{0.3\textwidth}\scriptsize
  \verb|\LaTeX ^_^|
  and
  \verb*`printf("Hello, world!\n");`
      \end{column}
  \end{columns}
\end{frame}

\subsection{交叉引用}

\begin{frame}[fragile]{交叉引用}

  \begin{itemize}
    \item  \LaTeX{} 中使用 \mintinline{tex}{\label} 标记,然后可以使用 \mintinline{tex}{\ref} 来引用这个标记。\mintinline{tex}{\label} 可以放在使用计数器的对象之后。
    \item 为了使得对公式编号的引用带有括号,推荐使用 AMSmath 宏包中的 \mintinline{tex}{\eqref} 命令。 对于多行公式环境,每一个换行符前都可以添加一个 \mintinline{tex}{\label} 用于引用该行公式。
  \end{itemize}
  \pause

  \begin{columns}
    \begin{column}{.4\textwidth} 
      \begin{minted}[frame=lines,framesep=2mm,baselinestretch=1.2,bgcolor=LightGray,fontsize=\scriptsize]{tex}
\begin{equation}
  a = b + c
\label{eq:example}
\end{equation}
\begin{figure}
  \includegraphics[width=3cm]{example-image-a}
  \caption{示例}\label{fig:example}
\end{figure}
如公式 \eqref{eq:example} 所示,
如图\ref{fig:example}所示
      \end{minted}
    \end{column}
    \begin{column}{.4\textwidth} \small
      \begin{equation}
        a = b + c
      \label{eq:example}
      \end{equation}
      \begin{figure}
        \includegraphics[width=2cm]{logo/logo.png}
        \caption{示例}\label{figure:example}
      \end{figure}
      如公式 \eqref{eq:example} 所示,
      如图\ref{figure:example}所示
    \end{column}
  \end{columns}
\end{frame}

\begin{frame}[fragile]{交叉引用}
  \begin{columns}
    \begin{column}{.5\textwidth} 
      \begin{minted}[frame=lines,framesep=2mm,baselinestretch=1.2,bgcolor=LightGray,fontsize=\scriptsize]{tex}
\begin{table} % \usepackage{booktabs}
  \caption{人员名单}
  \label{tab:example}
  \centering
  \begin{tabular}{cccccc}
    \toprule
    序号 & 姓名 & 性别 & 年龄 & 身高/cm & 体重/kg \\
    \midrule
    1 & 张三 & M & 16 & 163 & 50 \\
    2 & 王红 & F & 15 & 159 & 47 \\
    3 & 李二 & M & 17 & 165 & 52 \\
    \bottomrule
  \end{tabular}
\end{table}
      \end{minted}
    \end{column}
    \begin{column}{.4\textwidth} \tiny
      \begin{table}[]
        \caption{人员名单}
        \label{tab:example}
        \centering
        \begin{tabular}{cccccc}
          \toprule
          序号 & 姓名 & 性别 & 年龄 & 身高/cm & 体重/kg \\
          \midrule
          1 & 张三 & M & 16 & 163 & 50 \\
          2 & 王红 & F & 15 & 159 & 47 \\
          3 & 李二 & M & 17 & 165 & 52 \\
          \bottomrule
        \end{tabular}
      \end{table}
      如表\ref{tab:example} 所示
    \end{column}
  \end{columns}
\end{frame}

\subsection{文献管理}
\begin{frame}[fragile]
  \frametitle{文献管理}
  \begin{itemize}
    \item 建议自动生成\pause (你只有三篇参考文献?)\pause
    \item |.bib| 数据库
  
      \begin{itemize}
        \item Google Scholar 可直接复制:点击 \faQuoteRight \quad -> Bib\TeX{}
        \item 用 EndNote、Jabref 等生成
      \end{itemize} \pause
  
    \item 传统方法(大部分会议、期刊模板):Bib\TeX{}  后端
  
      \begin{itemize}
        \item 控制文献、引用样式:\pkg{natbib} 宏包
        \item 国家标准 GB/T 7714--2015
              \link{https://www.gb688.cn/bzgk/gb/newGbInfo?hcno=7FA63E9BBA56E60471AEDAEBDE44B14C}
              \link{https://github.com/Haixing-Hu/GBT7714-2005-BibTeX-Style/files/153951/GBT.7714-2015.pdf}:
              \alert{\pkg{gbt7714} 宏包}
      \end{itemize} \pause
  
    \item 现代方法:\pkg{biber} 后端 + \pkg{biblatex} 宏包
  
      \begin{itemize}
        \item 国家标准:\pkg{biblatex-gb7714-2015} 宏包
      \end{itemize} \pause
  
    \item 需多次编译
      \begin{itemize}
        \item pdf\LaTeX{} -> Bib\TeX{} -> pdf\LaTeX{} -> pdf\LaTeX{}
        \item Xe\LaTeX{} -> Bib\TeX{} -> Xe\LaTeX{} -> Xe\LaTeX{}
        \item 一键使用:\pkg{VS Code plugin}, \pkg{MakeFile}, \pkg{Batch} script, \pkg{latexmk}
      \end{itemize}
    
  \end{itemize}
\end{frame}

\begin{frame}[fragile]{文献引用样例}
  \begin{columns}
    \begin{column}{.6\textwidth} 
      \begin{minted}[frame=lines,framesep=2mm,baselinestretch=1.2,bgcolor=LightGray,fontsize=\scriptsize]{tex}
% In body.tex
“真理只有一个,而究竟谁发现了真理,不依靠主观的夸张,
而依靠客观的实践。”-- 毛泽东\cite{毛泽东1949新民主主义论}。

% In references.bib
@book{毛泽东1949新民主主义论,
  title={新民主主义论},
  author={毛泽东},
  year={1949},
  publisher={长江出版社}
}
      \end{minted}
    \end{column}
    \begin{column}{.4\textwidth}
      “真理只有一个,而究竟谁发现了真理,不依靠主观的夸张,而依靠客观的实践。” -- 毛泽东\cite{毛泽东1949新民主主义论}。

      \printbibliography[heading=none]
    \end{column}
    \end{columns}
\end{frame}
