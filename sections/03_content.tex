\section{Hello world案例}

\begin{frame}[fragile]{Hello World!}
  \begin{minted}
    [
      frame=lines,
      framesep=2mm,
      baselinestretch=1.2,
      bgcolor=LightGray,
    ]{tex}
% 用 pdfLaTeX、XeLaTeX 或 LuaLaTeX 编译
\documentclass{article}
\begin{document}
Hello world!
\end{document}
  \end{minted}
  \pause
  \begin{minted}
  [
    frame=lines,
    framesep=2mm,
    baselinestretch=1.2,
    bgcolor=LightGray,
  ]{tex}
% 用 XeLaTeX 或 LuaLaTeX 编译
\documentclass{ctexart}
\begin{document}
你好中国!
\end{document}
  \end{minted}
\end{frame}

\begin{frame}{语法}
  \begin{itemize}
      \item 命令(控制序列):以 \texttt{\textbackslash} 开头,大小写敏感
      \begin{itemize}
          \item \mintinline{tex}{\LaTeX{}} $\to$ \LaTeX{}
          \item 有作用域的概念,可以用 \texttt{\{\}} 限定作用域
          \item 可以接收参数,如 \mintinline{tex}{\documentclass{book}}
          \item 必需参数用 \texttt{\{\}},可选参数用 \texttt{[]}
      \end{itemize} 
      \pause
      \item 注释以 \texttt{\%} 开头
      \item 连续多个空格 = 单个空格 = 单个换行
      \item 可以定义自己的命令
  \end{itemize}
\end{frame}

\section{填写创作}

\begin{frame}[fragile]{文件的结构}
  \begin{minted}
    [
      frame=lines,
      framesep=2mm,
      baselinestretch=1.2,
      bgcolor=LightGray,
    ]{tex}
  \documentclass{article} % 文档类
  % 导言区 (preamble):加载宏包、定义命令、设置文档格式等
  
  \begin{document}        % 文档开始
  % 正文区 (body):文档的内容
  Hello world!
  \input{content.tex}     % 也可以导入其他文件内容
  \end{document}          % 文档结束,后续内容不会被编译
  \end{minted}
\end{frame}

\begin{frame}[fragile]{文档类与宏包}
  \begin{itemize}
      \item \LaTeX{} 文档开头必须包含 \mintinline{tex}{\documentclass} 指定文档类
      \item 基础文档类:\texttt{article}、\texttt{report}、\texttt{book}、\texttt{beamer} 等
      \item 中文变体:\texttt{ctexart}、\texttt{ctexrep}、\texttt{ctexbook}、\texttt{ctexbeamer} 等
      \item 可以通过可选参数指定字号、纸张大小、双面打印等选项
      \begin{minted}{tex}
          \documentclass[12pt, a4paper, twoside]{article}
      \end{minted}
      \pause
      \item 宏包相当于第三方库,可以引入更丰富的扩展功能
      \begin{minted}{tex}
          \usepackage{amsmath}
          \usepackage[final]{minted}
      \end{minted}
      \item \TeX{} Live full 自带大量宏包
      \item 通过 \texttt{texdoc \textit{package}} 查看文档
  \end{itemize}
\end{frame}

\section{段落及排版样式}

\begin{frame}[fragile]{段落 (paragraph)}
  \begin{itemize}
      \item 正常写作即可,以空行或 \mintinline{tex}{\par} 分段
      \item 行首空格会被忽略,多个空格会被合并为一个
      \item 换行在非中文语境下视作一个空格
  \end{itemize}
  
  \vspace{1em}
  \pause
  \begin{columns}
      \begin{column}{0.5\textwidth}
          \begin{minted}[frame=lines,framesep=2mm,baselinestretch=1.2,bgcolor=LightGray,fontsize=\scriptsize]{tex}
Consecutive spaces    are
squashed     in \LaTeX{}. 中文语境下
不会添加空格。Leading spaces are
    ignored. Trailing comment does%
n't break a word.

A new paragraph.\par
或者用 \verb|\par| 也可以。
          \end{minted}
      \end{column}
      \begin{column}{0.45\textwidth}
          \setlength{\parskip}{.5em}\small
  Consecutive spaces    are
  squashed     in \LaTeX{}. 中文语境下
  不会添加空格。Leading spaces are
      ignored. Trailing comment does%
  n't break a word.
  
  A new paragraph.\par
  或者用 \verb|\par| 也可以。
      \end{column}
  \end{columns}
\end{frame}
  
\begin{frame}[fragile]{断行与断页}
\begin{itemize}
    \item 可以用 \mintinline{tex}{\\} 或 \mintinline{tex}{\newline} 强制断行
    \begin{itemize}
        \item 类似 Word 中的``软回车'',不会产生新的段落
    \end{itemize}
    \item \mintinline{tex}{\newpage} 强制断页
    \item \mintinline{tex}{\clearpage} 强制断页并清空\emph{浮动体}
    \begin{itemize}
        \item We'll cover this soon ;)
        \item 一般用 \mintinline{tex}{\newpage} 即可
    \end{itemize}
\end{itemize}
\vspace{1em}
\pause

\begin{columns}
    \begin{column}{0.5\textwidth}
        \begin{minted}[frame=lines,framesep=2mm,baselinestretch=1.2,bgcolor=LightGray,fontsize=\scriptsize]{tex}
Starting a new paragraph. \\
Oops, a new line within it.

Finally another paragraph. \newline
Another new line, oh my!
        \end{minted}
    \end{column}
    \begin{column}{0.45\textwidth}
        \setlength{\parskip}{.5em}\small
Starting a new paragraph. \\
Oops, a new line within it.

Finally another paragraph. \newline
Another new line, oh my!
    \end{column}
\end{columns}
\end{frame}
  
\begin{frame}[fragile]{字形与字号}
\begin{itemize}
    \item 有两类修改字形的命令:\\\mintinline{tex}{{\bfseries bold}} 和 \mintinline{tex}{\textbf{bold}}
    \begin{itemize}
        \item 前者对作用域内的所有内容生效,后者只对参数内容生效
        \item 都是“内置”字体的不同样式,自定义字体?留作习题
    \end{itemize}
    \item 修改字号:\mintinline{tex}{{\large Hi}}
    \item \alert{汉字一般不使用斜体}
    \pause
    \item \alert{不要滥用字体样式}{\scriptsize(本页除外)} 
\end{itemize}
\pause
\begin{columns}
    \begin{column}{0.62\textwidth}
        \begin{minted}[frame=lines,framesep=2mm,baselinestretch=1.2,bgcolor=LightGray,fontsize=\scriptsize]{tex}
\textbf{bold} \textsf{sans serif}
\texttt{typewriter} \textsc{Small Caps}
\textit{italic} \textsl{slanted}

{\tiny tiny} {\scriptsize scriptsize}
{\footnotesize footnotesize} {\small small}
{\normalsize normalsize} {\large large}
{\Large Large} {\LARGE LARGE}
{\huge huge} {\Huge Huge}
        \end{minted}
    \end{column}
    \begin{column}{0.39\textwidth}\vspace{-2em}

\textbf{bold} \textsf{sans serif}
\texttt{typewriter} \textsc{Small Caps}
\textit{italic} {\addfontfeatures{AutoFakeSlant}\textsl{slanted}}

{\tiny tiny} {\scriptsize scriptsize}
{\footnotesize footnotesize} {\small small}
{\normalsize normalsize} {\large large}
{\Large Large} {\LARGE LARGE}
{\huge huge} {\Huge Huge}
    \end{column}
\end{columns}
\end{frame}
  
\begin{frame}[fragile]{关于字符和标点}
\begin{itemize}
    \item 特殊字符需要转义,如 \mintinline{tex}{\_}、\mintinline{tex}{\%}、\mintinline{tex}{\$}、\mintinline{tex}{\&}、\mintinline{tex}|\{|、\mintinline{tex}|\}| 等等
    \begin{itemize}
        \item \mintinline{tex}{\~{}}、\mintinline{tex}{\^{}} 带大括号防止将后面的字符理解为参数
        \item \mintinline{tex}{\textbackslash} 反斜杠
    \end{itemize}
    \item 西文语境下的引号:\mintinline{tex}{``double''} \quad ``double''
    \begin{itemize}
        \item 中文语境下正常输入即可
    \end{itemize}
    \item 连字符:\mintinline{tex}{-}、短破折号:\mintinline{tex}{--}、长破折号:\mintinline{tex}{---}
    \begin{itemize}
        \item Newton-Leibniz Formula
        \item Henry Kissinger (1923--2023)
        \item Don't use \texttt{\$\$} --- it breaks spacing!
    \end{itemize}
    \item 防止西文连字:\mintinline{tex}{dif{}f{}icult} \quad dif{}f{}icult
    \begin{itemize}
        \item 对比:difficult
    \end{itemize}
\end{itemize}
\end{frame}
  
\begin{frame}[fragile]{强制行距/间距}
\begin{itemize}
    \item 长度单位:pt,in(英寸),cm,mm,em(小写 `m' 的宽度),ex(小写 `x' 的高度)
    \item 行距:\mintinline{tex}{\linespread{factor}}(默认行距 1.2 倍字号大小)
        \begin{itemize}
            \item 局部修改要在后面加 \mintinline{tex}{\selectfont},且要在范围内手动分段
        \end{itemize}
    \item 插入水平间距(多空格无效):\mintinline{tex}{\hspace{length}}
        \begin{itemize}
            \item \mintinline{tex}{\quad} 相当于 1em,\mintinline{tex}{\qquad} 相当于 2em
            \item \mintinline{tex}{\hspace*} 防止因为断行而消失
        \end{itemize}
    \item 插入垂直间距:\mintinline{tex}{\vspace{length}}
        \begin{itemize}
            \item \mintinline{tex}{\\[1em]} 可以在换行的同时插入垂直间距
        \end{itemize}
\end{itemize}
\pause
\begin{columns}
    \begin{column}{0.5\textwidth}
        \begin{minted}[frame=lines,framesep=2mm,baselinestretch=1.2,bgcolor=LightGray,fontsize=\tiny]{tex}
{\linespread{1.6}\selectfont This is a paragraph with 1.6 linespread.\\
This is the next line and \quad space and \hspace{3em} more and
\hspace{\fill}lol\par}

\vspace{-1em}Next paragraph.\\[1em] and next line.
        \end{minted}
    \end{column}
    \begin{column}{0.45\textwidth}
        \setlength{\parskip}{.5em}\scriptsize
{\linespread{1.6}\selectfont This is a paragraph with 1.6 linespread.\\
This is the next line and \quad space and \hspace{3em} more and
\hspace{\fill}lol\par}

\vspace{-1em}Next paragraph.\\[1em] and next line.
    \end{column}
\end{columns}
\end{frame}
  
\begin{frame}[fragile]{标题页}
    在导言区:
\begin{itemize}
    \item \mintinline{tex}{\title}\texttt{\small\{\textit{title}\}}:标题
    \item \mintinline{tex}{\author}\texttt{\small\{\textit{author}\}}:作者
    \item \mintinline{tex}{\date}\texttt{\small\{\textit{date}\}}:日期,不写则默认为当前日期 \mintinline{tex}{\today}
\end{itemize}
    在正文区使用 \mintinline{tex}{\maketitle} 生成标题页
\pause
\begin{columns}
    \begin{column}{0.4\textwidth}
        \centering
        \begin{minted}[frame=lines,framesep=2mm,baselinestretch=1.2,bgcolor=LightGray,fontsize=\scriptsize]{tex}
\title{\textbf{我是标题}}
\author{箱子不知道哦}
\date{1969 年 12 月 31 日}
\begin{document}
\maketitle
\end{document}
        \end{minted}
    \end{column}
    \begin{column}{0.45\textwidth}
        \centering
        \includegraphics[width=0.5\textwidth]{images/titlepage.pdf}
    \end{column}
\end{columns}
\end{frame}
  
\begin{frame}[fragile]{环境}
\begin{itemize}
    \item 成对出现的命令,用于控制一段内容的格式
    \item 以 \mintinline{tex}{\begin{environment}} 开始,以 \mintinline{tex}{\end{environment}} 结束
    \item 常见环境:\texttt{itemize}、\texttt{figure}、\texttt{table} 等
\end{itemize}
\vspace{1em}
\pause
\begin{columns}
    \begin{column}{0.4\textwidth}
        \begin{minted}[frame=lines,framesep=2mm,baselinestretch=1.2,bgcolor=LightGray,fontsize=\scriptsize]{tex}
\begin{itemize}
    \item macOS
    \item Windows
    \item Linux
\end{itemize}
        \end{minted}
    \end{column}
    \begin{column}{0.4\textwidth}
        \small
\begin{itemize}
    \item macOS
    \item Windows
    \item Linux
\end{itemize}
    \end{column}
\end{columns}
\end{frame}

\subsection{目录}

\begin{frame}{章节和目录}
\begin{itemize}
    \item 应将文档合理地分割为章、节、小节等
    \begin{itemize}
        \item \mintinline{tex}{\part}、\mintinline{tex}{\chapter} 仅适用于 \texttt{book} 和 \texttt{report} 文档类
        \item \mintinline{tex}{\section}、\mintinline{tex}{\subsection}、\mintinline{tex}{\subsubsection}
        \item 不常用:\mintinline{tex}{\paragraph}、\mintinline{tex}{\subparagraph}
    \end{itemize}
    \item 自动编号,默认编号深度为 3,可以修改
    \item 带星号的版本不会自动编号,也不会出现在目录中(\mintinline{tex}{\section*})
    \pause
    \item \mintinline{tex}{\tableofcontents} 生成目录
    \begin{itemize}
        \item 如果不使用 \texttt{latexmk},需要多次编译
    \end{itemize}
    \item \mintinline{tex}{\appendix} 起为附录,后续章节编号为大写字母
\end{itemize}
\end{frame}

\begin{frame}[fragile]{层次与目录生成}
    \begin{columns}
        \begin{column}{0.4\textwidth}
            \begin{minted}[frame=lines,framesep=2mm,baselinestretch=1.2,bgcolor=LightGray,fontsize=\scriptsize]{tex}
\tableofcontents % 这里是目录
\part{有监督学习}
\chapter{支持向量机}
\section{支持向量机简介}
\subsection{支持向量机的历史}
\subsubsection{支持向量机的诞生}
\paragraph{一些趣闻}
\subparagraph{第一个趣闻}
            \end{minted}
        \end{column}
        \begin{column}{0.4\textwidth}
            第一部分\quad 有监督学习\\
            第一章\quad 支持向量机 \\
            1. 支持向量机简介 \\
            1.1 支持向量机的历史 \\
            1.1.1 支持向量机的诞生 \\
            一些趣闻  \\
            第一个趣闻
        \end{column}
    \end{columns}
\end{frame}

\subsection{图片}

\begin{frame}[fragile]{图片}
  \begin{itemize}
      \item 需要 \texttt{graphicx} 宏包{\small{}(注意不是 \texttt{graphic\textbf{s}})}
      \item \mintinline{tex}{\includegraphics}\texttt{\small[\textit{options}]\{\textit{filename}\}}
      \begin{itemize}
          \item 常用选项:\texttt{width}、\texttt{height}、\texttt{scale} 宽高缩放
          \item \texttt{\textit{filename}} 可以是相对路径或绝对路径
      \end{itemize}
  \end{itemize}
  \vspace{1em}
  \pause
  \begin{columns}
      \begin{column}{0.55\textwidth}
          \begin{minted}[highlightlines={4,8,9},frame=lines,framesep=2mm,baselinestretch=1.2,bgcolor=LightGray,fontsize=\scriptsize]{tex}
  \usepackage{graphicx}
  \graphicspath{{lec4/}}
  % ...
  \begin{figure}
    \centering
    \includegraphics[...]{logo/logo.png}
    \caption{上海电力大学校徽}
    \label{fig:suep:LOGO}
  \end{figure}
          \end{minted}
      \end{column}
      \begin{column}{0.45\textwidth}
          \scriptsize
  \begin{figure}
      \centering
      \includegraphics[width=.6\textwidth]{logo/logo.png}
      \caption{上海电力大学校徽}
      \label{fig:suep:LOGO}
  \end{figure}
      \end{column}
  \end{columns}
\end{frame}
  
\begin{frame}[fragile]{浮动体}
  \begin{itemize}
      \item 再次强调:\alert{别把 \LaTeX{} 当 Word 用}
      \begin{minted}[fontsize=\scriptsize,highlightlines={1,3}]{tex}
  \begin{figure}[htbp]
      % ...
  \end{figure}
      \end{minted}
      \item 摆脱“见下图”、“如上图所示”、“如表所示”……
      \item \mintinline{tex}{[htbp]} 是浮动体的位置选项:here / top / bottom / page
      \item 使用\alert{交叉引用}:\mintinline{tex}{\label} 和 \mintinline{tex}{\ref}、\mintinline{tex}{\pageref}、\mintinline{tex}{\nameref}、\mintinline{tex}{\eqref}
      \item 进阶:\texttt{hyperref} 与 \texttt{cleveref} 等宏包
  \end{itemize}
  \pause
  \begin{minted}{tex}
  图 \ref{fig:ckc} 是上海电力大学校徽
  (在 PDF 中是 \pageref{fig:ckc} 页)。
  \end{minted}
  图 \ref{fig:ckc} 是上海电力大学校徽
  (在 PDF 中是 \pageref{fig:ckc} 页)。
  
  推荐阅读:\href{http://static.latexstudio.net/wp-content/uploads/2018/03/LianTze-presentation-0320-forReading.pdf}{漫谈 \LaTeX{} 排版常见概念误区}
\end{frame}
  
\begin{frame}[fragile]{设置页大小及页边距}
  \begin{itemize}
      \item \texttt{geometry} 宏包
      \item 全局分栏:\mintinline{tex}{\twocolumn} 和 \mintinline{tex}{\onecolumn}
      \item 局部分栏:\texttt{multicol} 宏包的 \texttt{multicols} 环境
  \end{itemize}
  \begin{minted}[frame=lines,framesep=2mm,baselinestretch=1.2,bgcolor=LightGray]{tex}
\usepackage{geometry}

\geometry{paper=b5paper}        % 标准纸张大小
\geometry{paperwidth=260mm,
          paperheight=185mm}    % 自定义纸张大小
\geometry{margin=1in}           % 全相等
\geometry{left=1in, right=1in,
          top=1in, bottom=1in}  % 分别设置
\geometry{hmargin=1.5in, vmargin=1in}   % 水平和垂直
  \end{minted}
\end{frame}
  
\subsection{列表}

\begin{frame}[fragile]{列表}
  \begin{itemize}
      \item \texttt{itemize}:无序列表
      \item \texttt{enumerate}:有序列表
      \item \texttt{description}:描述列表
      \item 列表中的每一项都是一个 \mintinline{tex}{\item}
      \begin{itemize}
          \item 可选参数可以修改编号样式
      \end{itemize}
      \item 推荐使用 \texttt{enumitem} 宏包
  \end{itemize}
  \pause
  \begin{columns}
      \begin{column}{0.55\textwidth}
          \begin{minted}[frame=lines,framesep=2mm,baselinestretch=1.2,bgcolor=LightGray,fontsize=\scriptsize]{tex}
  \begin{description}
      \item[Slackware] 历史最悠久
      \item[Debian] 历史第二悠久
      \item[Ubuntu] 基于 Debian,最流行
      \item[Kali Linux] 面向安全工作者
  \end{description}
          \end{minted}
      \end{column}
      \begin{column}{0.45\textwidth}\scriptsize
  \begin{description}
      \item[Slackware] 历史最悠久
      \item[Debian] 历史第二悠久
      \item[Ubuntu] 基于 Debian,最流行
      \item[Kali Linux] 面向安全工作者
  \end{description}
      \end{column}
  \end{columns}
  \end{frame}
  
  \begin{frame}[fragile]{对齐环境}
  \begin{itemize}
      \item \texttt{center} 居中,\texttt{flushleft} 左对齐,\texttt{flushright} 右对齐
      \begin{itemize}
          \item 这里指的是对齐\emph{环境}
          \item 会在环境上下额外生成间距
      \end{itemize}
      \item \mintinline{tex}{\centering}、\mintinline{tex}{\raggedright}、\mintinline{tex}{\raggedleft}
      \begin{itemize}
          \item 这里指的是对齐\emph{命令}
          \item 不会生成额外间距,直接改变对齐方式
          \item 注意左对齐是 \mintinline{tex}{\raggedright}
      \end{itemize}
  \end{itemize}
  \pause
  \begin{columns}
      \begin{column}{0.4\textwidth}
          \begin{minted}[frame=lines,framesep=2mm,baselinestretch=1.2,bgcolor=LightGray,fontsize=\scriptsize]{tex}
  \begin{center}
      some text
  \end{center}
  \begin{flushright}
      some text
  \end{flushright}
  
  \centering some text \par
  \raggedleft some text
          \end{minted}
      \end{column}
      \begin{column}{0.3\textwidth}\scriptsize
  \begin{center}
      some text
  \end{center}
  \begin{flushright}
      some text
  \end{flushright}
  
  \centering some text \par
  \raggedleft some text
      \end{column}
  \end{columns}
  \end{frame}

% \begin{frame}[fragile]{\LaTeX{} 环境举例}
%     \vspace{1em}
%     \begin{minipage}{0.4\linewidth}
%         \begin{lstlisting}[basicstyle=\ttfamily\small]
% \begin{itemize}
%     \item 一条
%     \item 次条
%     \item 这一条可以分为 ...
%     \begin{itemize}
%         \item 子一条
%     \end{itemize}
% \end{itemize}
%          \end{lstlisting}
%     \end{minipage}\hspace{1.5cm}
%     \begin{minipage}{0.4\linewidth}
%         \begin{itemize}
%             \item 一条
%             \item 次条
%             \item 这一条可以分为 ...
%             \begin{itemize}
%                 \item 子一条
%             \end{itemize}
%         \end{itemize}
%     \end{minipage}
% % \smallskip

% \begin{minipage}{0.4\linewidth}
% \begin{lstlisting}
% \begin{enumerate}
%     \item 一条
%     \item 次条
%     \item 再条
% \end{enumerate}\end{lstlisting}
%     \end{minipage}\hspace{1.5cm}
%     \begin{minipage}{0.4\linewidth}
%     \vspace{-1cm}
% \begin{enumerate}
%     \item 一条
%     \item 次条
%     \item 再条
% \end{enumerate}
%     \end{minipage}
% \end{frame}
%     %
    
% \begin{frame}[fragile]{列表与枚举}
% \begin{columns}
% \begin{column}{.6\textwidth}
% \begin{lstlisting}[basicstyle=\ttfamily\small]
% \begin{enumerate}
% \item \LaTeX{} 好处都有啥
% \begin{description}
%     \item[好用] 体验好才是真的好
%     \item[好看] 强迫症的福音
%     \item[开源] 众人拾柴火焰高
% \end{description}
% \item 还有呢?
% \begin{itemize}
%     \item 好处 1
%     \item 好处 2
% \end{itemize}
% \end{enumerate}
% \end{lstlisting}
% \end{column}
% \begin{column}{.4\textwidth}
% {\small
% \begin{enumerate}
% \item \LaTeX{} 好处都有啥
%     \begin{description}
%     \item[好用] 体验好才是真的好
%     \item[好看] 治疗强迫症
%     \item[开源] 众人拾柴火焰高
%     \end{description}
% \item 还有呢?
%     \begin{itemize}
%     \item 好处 1
%     \item 好处 2
%     \end{itemize}
% \end{enumerate}
% }
% \end{column}
% \end{columns}

% \end{frame}

% \subsection{目录}
%     \begin{frame}[fragile]{层次与目录生成}
%     \begin{columns}
%     \begin{column}{.6\textwidth}
%     \lstset{language=[LaTeX]TeX}
%     \begin{lstlisting}[basicstyle=\ttfamily\small]
% \tableofcontents % 这里是目录
% \part{有监督学习}
% \chapter{支持向量机}
% \section{支持向量机简介}
% \subsection{支持向量机的历史}
% \subsubsection{支持向量机的诞生}
% \paragraph{一些趣闻}
% \subparagraph{第一个趣闻}
%     \end{lstlisting}
%     \end{column}
%     \begin{column}{.4\textwidth}
%     第一部分\quad 有监督学习\\
%     第一章\quad 支持向量机 \\
%     1. 支持向量机简介 \\
%     1.1 支持向量机的历史 \\
%     1.1.1 支持向量机的诞生 \\
%     一些趣闻  \\
%     第一个趣闻
%     \end{column}
%     \end{columns}
    
%     \end{frame}

%     \subsection{插图,表格,交叉引用}
%     \begin{frame}[fragile]{交叉引用与插入插图}
%         \begin{itemize}
%         \item 给对象命名:图片、表格、公式等\\
%         |\label{name}|
%       \item 引用对象\\
%         |\ref{name}|
%         \end{itemize}
%       \bigskip
      
%         \begin{minipage}{0.7\linewidth}
%           \lstset{language=[LaTeX]TeX}
%           \begin{lstlisting}
% 上海电力大学校徽请参见图~\ref{fig:sustech:LOGO}。
% \begin{figure}[htbp]
%   \centering
%   \includegraphics[height=.2\textheight]%
%   {LOGO.png}
%   \caption{上海电力大学校徽。}
%   \label{fig:sustech:LOGO}
% \end{figure}
%       \end{lstlisting}
%         \end{minipage}\hfill
%         \begin{minipage}{0.3\linewidth}\centering
%           {\songti 上海电力大学校徽请参见图~1。}\\[1em]
%        \includegraphics[height=0.2\textheight]{logo.png}\\
%        {\footnotesize\heiti 图~1. 上海电力大学校徽。}
%         \end{minipage}
%       \end{frame}
      
%       \begin{frame}[fragile]{交叉引用与插入表格}
%         \begin{columns}
%         \column{.6\textwidth}
%         \lstset{language=[LaTeX]TeX}
%         \begin{lstlisting}
% \begin{table}[htbp]
%     \caption{编号与含义}
%     \label{tab:number}
%     \centering
%     \begin{tabular}{cl}
%       \hline
%       编号 & 含义 \\
%       \hline
%       1    & 第一 \\
%       2    & 第二 \\
%       \hline
%     \end{tabular}
% \end{table}
% 公式~(\ref{eq:vsphere}) 中编号与含义
% 请参见表~\ref{tab:number}。
%       \end{lstlisting}
%       \column{.4\textwidth}
%       \centering
%       {\small 表~1. 编号与含义}\\[2pt]
%       \begin{tabular}{cl}\hline
%       编号 & 含义 \\\hline
%       1 & 第一\\
%       2  & 第二\\\hline
%       \end{tabular}\\[5pt]
      
%       \normalsize 公式~(\ref{eq:vsphere})编号与含义请参见表~1。
%         \end{columns}
%       \end{frame}
      
%       \begin{frame}[fragile]{浮动体}
%       \begin{itemize}
%       \item 初学者最“捉摸不透”的特性之一 \url{https://liam.page/2017/03/11/floats-in-LaTeX-basic}
%       \item 图片和表格有时会很大,在插入的位置不一定放得下,因此需要浮动调整
%       \item 避免在文中使用「下图」「上图」的说法,而是使用图表的编号,例如 |图~\ref{fig:fig1}| 。
%       \item |\begin{figure}[<位置>] 图片 \end{figure}|
%         \begin{itemize}
%         \item 位置参数指定浮动体摆放的偏好
%         \item |h| 当前位置(here), |t| 顶部(top), |b| 底部(bottom), |p| 单独成页(p)
%         \item |!h| 表示忽略一些限制,|H| 表示强制\alert{(强烈不建议,除非你知道自己在做什么)}
%         \end{itemize}
%       \item 温馨提示:图标题一般在下方,表标题一般在上方
%       \end{itemize}
%       \end{frame}
      
%       \begin{frame}[fragile]
%         \frametitle{作图与插图}
%         \begin{itemize}
%           \item 外部插入
      
%             \begin{itemize}
%               \item Mathematica、MATLAB
%               \item PowerPoint、Visio、Adobe Illustrator、Inkscape
%               \item Python \pkg{Matplotlib} 库、\texttt{Plots.jl}、R、Plotly 等
%               \item draw.io \url{https://draw.io/}、ProcessOn \url{https://www.processon.com/} 等在线绘图网站
%             \end{itemize}
      
%           \item \TeX 内联
      
%             \begin{itemize}
%               \item Asymptote
%               \item \alert{\pkg{pgf}/\pkg{TikZ}、\pkg{pgfplots}}
%             \end{itemize}
      
%           \item 插图格式
      
%             \begin{itemize}
%               \item 矢量图:|.pdf| 或 |.eps|
%               \item 位图:|.jpg| 或 |.png|
%               \item 不(完全)支持 |.svg|、|.bmp|
%             \end{itemize}
      
%           \item 参考:如何在论文中画出漂亮的插图?\link{https://www.zhihu.com/question/21664179}
%         \end{itemize}
%       \end{frame}
      
%       \begin{frame}[fragile]{表格绘制}
%         \begin{itemize}
%           \item 使用 \pkg{booktabs}(三线表)、\pkg{longtables}(跨页表)、\pkg{multirow}(单元格内换行) 等宏包
%           \item 手动绘制表格确实比较令人头疼,且较难维护
%           \item 推荐使用在线工具绘制后导出代码:
%             \begin{itemize}
%               \item \LaTeX{} Tables Editor \link{https://www.latex-tables.com/}
%               \item \LaTeX{} Table Generator \link{https://www.tablesgenerator.com/latex_tables}
%             \end{itemize}
%         \end{itemize}
%       \end{frame}
      


