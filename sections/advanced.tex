\section{进阶}
\subsection{beamer}
\begin{frame}[fragile]
  \frametitle{幻灯片}
  \begin{itemize}
    \item<+-> 基本框架
  
      \begin{itemize}
        \item \pkg{beamer} 或 \pkg{ctexbeamer} 文档类
        \item 页面由 \mintinline{tex}{frame}  环境组织
        \item 文本内容:建议使用 \mintinline{tex}{itemize} 和 \mintinline{tex}{enumerate}
        \item 图表:不再浮动,不建议使用交叉引用
        \item 定理及强调:\mintinline{tex}{theorem} 、\mintinline{tex}{proof} 、\mintinline{tex}{block} 等
        \item 分栏:\mintinline{tex}{columns} + \mintinline{tex}{columns} 环境
      \end{itemize}
  
    \item<+-> 主题与样式
  
      \begin{itemize}
        \item \mintinline{tex}{\usetheme}、\mintinline{tex}{\lstinline[style=style@inline]+\use[font|color|inner|outer]theme+}
        \item 更现代的主题:\mintinline{tex}{metropolis}
        \item 使用「默认」字体:\mintinline{tex}{\usefonttheme{serif}}
      \end{itemize}
  
    \item<+-> 动画(覆盖)
  
      \begin{itemize}
        \item \mintinline{tex}{\pause}命令
        \item \mintinline{tex}{\onslide<1>}、\mintinline{tex}{\item<1->}等
      \end{itemize}
  \end{itemize}
\end{frame}
  
\subsection{git}

\begin{frame}[fragile]
  \frametitle{Git版本管理}
  \begin{itemize}
    \item<+-> 版本管理的必要性
      \begin{itemize}
        \item 远离「初稿,第二稿……终稿,终稿(打死也不改了)」命名
        \item 有底气做大范围修改、重构
        \item 方便与他人协同合作
      \end{itemize}
    \item<+-> 基本用法
      \begin{itemize}
        \item 跟踪更改:\verb|git init|、\verb|git add|
              \verb|git commit|
        \item 撤销与回滚:\verb|git reset|、\verb|git revert|
        \item 分支与高级用法:\verb|git branch|、\verb|git checkout|
              \verb|git rebase|
        \item 远端仓库操作:\verb|git pull|、\verb|git push|、
              \verb|git fetch|
        \item 推荐用 VS Code 等进行可视化操作
        \item 参考链接:\link{https://git-scm.com/book/en/v2}
              \link{https://www.liaoxuefeng.com/wiki/0013739516305929606dd18361248578c67b8067c8c017b000}
      \end{itemize}

    \item<+-> GitHub \href{https://github.com}{\faGithub} \& more

      \begin{itemize}
        \item 远程 Git 仓库
        \item Clone \& fork
        \item Issues \& pull requests
        \item<+-> \alert{提醒:绑定 \texttt{.edu} 邮箱可以有更多优惠}
      \end{itemize}
  \end{itemize}
\end{frame}
